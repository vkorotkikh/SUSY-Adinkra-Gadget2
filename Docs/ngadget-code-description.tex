\documentclass[12pt, letterpaper]{article}
\usepackage[utf8]{inputenc}


\title{New Gadget Code Description}
\author{Vadim Korotkikh}
\date{November 2017}

\begin{document}

\maketitle

\begin{abstract}
This document contains the medium level detail description of the python code
for generating 36k Adinkras and New Gadget values. Brief intro simple paragraph
at the beginning of the document.
\end{abstract}

This document write up contains the description of the software code algorithm
that is used to calculate the New Gadget for the entire BC4 Coxeter group space
of 36,864 Adinkras and how it is written and executed using Python 3.
Specific Python version used in calculation was Python 3.5, but the code is also
compatible with Python 2.7.
Software wise the code builds upon earlier developments/works by the author but
with changes to the code that pertains to the final gadget calculation.
To speed up Gadget calculation, multiprocessing feature has been added and is
utilized within the code. The code also now produces a text output of the
results which can be zip compressed for distribution/sharing of results.\par
Utilizing the BC4 Coxeter group space, the adinkra{\_}nxn{\_}constructor.py
code creates the 384 L sign permutation matrices. These L sign matrices serve as the building
blocks of all Adinkras, given that any two of them satisfy conditions of a set Garden algebra
equations. Once the 384 L matrices are created, for each L matrix the script builds a list of
compatible matrices that satisfy Garden Algebra conditions. For 384 L matrices, counting all
possible matrix position permutations, there is a grand total of 36,864 Adinkras with N=4 or
four color, four open node and four close node.\par
For calculating the Fermionic Holoraumy matrices the script fx{\_}vij{\_}holoraumy.py is used.
This script takes the input of 36,864 Adinkras and for each Adinkra generates a set of six
V Holoraumy matrices.
\end{document}
