\documentclass[12pt, letterpaper]{article}
\usepackage[utf8]{inputenc}


\title{New Gadget Code Description}
\author{Vadim Korotkikh}
\date{November 2017}

\begin{document}

\maketitle

\begin{abstract}
This document contains the medium level detail description of the python code
for generating 36k Adinkras and New Gadget values. Brief intro simple paragraph
at the beginning of the document.
\end{abstract}

This document write up contains the description of the software code algorithm
that is used to calculate the New Gadget for the entire BC4 Coxeter group space
of 36,864 Adinkras and how it is written and executed using Python 3.
Specific Python version used in calculation was Python 3.5, but the code is also
compatible with Python 2.7.
Software wise the code builds upon earlier developments/works by the author but
with changes to the code that pertains to the final gadget calculation.
To speed up Gadget calculation, multiprocessing feature has been added and is
utilized within the code. The code also now produces a text output of the
results which can be zip compressed for distribution/sharing of results. \par


Using the elemets of BC4 Coxeter group, the code \emph{adinkra_nxn_constructor}
creates the 384 L sign permutation matrices (4x4 matrices) These L sign matrices
serve as the building blocks of all Adinkras, given that any two of them satisfy
conditions of a set Garden algebra equations . The same
script also handles the process of assembling all possible Adinkras using the L
sign permutation matrices. Starting with 384 L sign matrices, for each L matrix
the script performs a cross reference loop check that builds a total of 36,864
Adinkras with N=4 (4 colors), four open-nodes and four closed nodes.

For calculating the Fermionic Holoraumy matrices the script fx_vij_holoraumy.py
is used. The actual function to do so is calc_fermi_vij( adinkra list) and it
takes in a ‘list’ or an array of pointers to arrays containing 4 L matrices
defined using numpy np.array. Using the 4 L sign permutation matrices, the
V~tilde Holoraumy matrices, V12, V13, V14, V23, V24, V34 are calculated for each
Adinkra. Each set of six V~tilde Holoraumy matrices are appended to a list which
then gets appended/added to the final return list. The returned array/list called
vij_fermi is passed back to it’s execution origin in cls_adinkra.py script which
serves to declare and describe AdinkraSet class. AdinkraSet class is used to
executing various Holoraumy functions and storing the return results to be used
by other functions.
\end{document}
