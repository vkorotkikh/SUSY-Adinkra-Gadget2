\documentclass[12pt, letterpaper]{article}
\usepackage[utf8]{inputenc}
\usepackage{hyperref}
\usepackage{graphicx}

\def\brV{{\bm {\rm V}}}
\def\brtV{{\bm{\tilde{\rm V}}}}

\title{New Gadget Code Description}
\author{Vadim Korotkikh}
\date{November 2017}

\begin{document}

\maketitle

\begin{abstract}
This document contains the medium level detail description of the python code
for generating 36k Adinkras and New Gadget values. Brief intro simple paragraph
at the beginning of the document.
\end{abstract}

This document write up contains the description of the software code algorithm
that is used to calculate the New Gadget for the entire BC4 Coxeter group space
of 36,864 Adinkras and how it is written and executed using Python 3.
Specific Python version used in calculation was Python 3.5, but the code is also
compatible with Python 2.7.
Software wise the code builds upon earlier developments/works by the author but
with changes to the code that pertains to the final gadget calculation.
To speed up Gadget calculation, multiprocessing feature has been added and is
utilized within the code. The code also now produces a text output of the
results which can be zip compressed for distribution/sharing of results.\par
Utilizing the BC4 Coxeter group space, the adinkra{\_}nxn{\_}constructor.py
code creates the 384 L sign permutation matrices. These L sign matrices serve as the building
blocks of all Adinkras, given that any two of them satisfy conditions of a set Garden algebra
equations. Once the 384 L matrices are created, for each L matrix the script builds a list of
compatible matrices that satisfy Garden Algebra conditions. For 384 L matrices, counting all
possible matrix position permutations, there is a grand total of 36,864 Adinkras with N=4 or
four color, four open node and four close node.\par
For calculating the Fermionic Holoraumy matrices the
fx{\_}vij{\_}holoraumy.py script is used.
This script takes the input of 36,864 Adinkras and for each Adinkra generates a set of six
V Holoraumy matrices. This script can generate the $\brtV$ Fermionic Holoraumy matrices and the
$\brV$ Bosonic Holoraumy matrices as well as calculate the elle coefficients for $\brtV$ matrices.
This depends on the calculation configuration/setting used in run{\_}adinkra{\_}calc.py
script. Within fx{\_}vij{\_}holoraumy.py there are three functions responsible for handling these tasks.
The fermionic{\_}holomats function calculates a set of six $\brtV$ matrices for a
given Adinkra. The bosonic{\_}holomats function calculates the set of six $\brV$ matrices.
The calc{\_}vij{\_}alphabeta function uses the six $\brtV$ matrices and alpha beta matrices
to calculate the corresponding $\tilde{\ell}$ coefficient values which are used later
in the calculation of Adinkra x Adinkra Gadget values. Once the calculation is finished
executing the fx{\_}vij{\_}holoraumy.py script returns an array (list in Python) of calculated
Holoraumy matrices or $\tilde{\ell}$ coefficients.\par
The New Gadget value calculation is done by fx{\_}mpgadgets.py script. This script utilizes
the multiprocessing features available in Python and this is reflected as 'mp' in the script name.
The task of calculating Adinkra Gadget values is cumbersome because each Gadget calculation
takes two Adinkras (they can be the same) this means there is a total of 1.36 billion possible
Gadget calculations. Therefore multiprocessing was used as a way to significantly speed up
Adinkra calculations over the BC4 space.

\end{document}
